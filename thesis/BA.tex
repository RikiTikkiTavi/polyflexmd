\documentclass[
    paper=A4,pagesize=automedia,fontsize=12pt,
    BCOR=15mm,DIV=22,
    twoside,headinclude,footinclude=false,
    ngerman,fleqn,             % fleqn = linksbündige Ausrichtung von Formeln
    bibliography=totocnumbered,          % Literaturverz. im Inhaltsverz. eintragen
    listof=totoc,                % Abbildungsverz. im Inhaltsverz. eintragen
    listof=flat,                 % Abbildungsverz. an der längsten Nummer ausrichten
    cleardoublepage=empty      % Vakatseiten ohne Paginierung
    numbers=endperiod
]{scrartcl}
\setlength\parindent{0em}

\usepackage{hyperref}
\hypersetup{
    colorlinks=false, %set true if you want colored links
    linktoc=all,     %set to all if you want both sections and subsections linked
    linkcolor=blue,  %choose some color if you want links to stand out
}

% Kodierung, Schrift und Sprache auswählen
\usepackage[utf8]{inputenc}
\usepackage[T1]{fontenc}
\usepackage[english]{babel}
% damit man Text aus dem PDF korrekt rauskopieren kann
\usepackage{cmap}
% Layout: Kopf-/Fußzeilen, anderthalbfacher Zeilenabstand
\usepackage[headsepline]{scrlayer-scrpage}
\pagestyle{scrheadings}
\clearpairofpagestyles
\ihead{\headmark}
\ohead{\pagemark}
\automark[subsection]{section}
\KOMAoption{headsepline}{0.5pt}

\usepackage{setspace}
\onehalfspacing
\deffootnote{1em}{1em}{\textsuperscript{\thefootnotemark}}

% Grafiken, Tabellen, Mathematikumgebungen
\usepackage{graphicx,xcolor}
\usepackage{tabularx}
\usepackage{amsmath,amsfonts,amssymb}

% Darstellung von Fließumgebungen
\usepackage{flafter,afterpage}
\usepackage[section]{placeins}
\usepackage[margin=8mm,font=small,labelfont=bf,format=plain]{caption}
\usepackage[margin=8mm,font=small,labelfont=bf,format=plain]{subcaption}

% \numberwithin{equation}{chapter}
% \numberwithin{figure}{chapter}
% \numberwithin{table}{chapter}

%%%%%%%%%%%%%%%%%%%%%%%%%%%%%%%%%%%%%%%%%%%%%%%%%%%%%%%%%%%%%%%%%%%%%%%%%%%%%%%%
% Ab hier ist Platz für eigene Ergänzungen (Pakete, Befehle, etc.)

% Dieses Paket liefert den Blindtext, der als Platzhalter in den Beispieldateien steht.
% Das kannst Du also entfernen, wenn Du den Blindtext nicht mehr brauchst.
\usepackage{lipsum}

\begin{document}

% \frontmatter

% Commands
\newcommand{\E}[1]{\langle#1\rangle}

% Titelpageseite
\begin{titlepage}
    \begin{tabularx}{\linewidth}{X}
        \includegraphics[width=6cm]{img/TU_Logo_SW} \\\hline\hline

        \vspace{4.5em}

        \begin{singlespace}\begin{center}\bfseries\Huge

            Molecular dynamics study of ideal polymer chains with different persistence lengths

            \end{center}\end{singlespace}

        \vspace{5.5em}

        \begin{singlespace}\begin{center}\large
                Bachelor-Arbeit \\ zur Erlangung des Hochschulgrades \\
                Bachelor of Science \\
                im Bachelor-Studiengang Physik
            \end{center}\end{singlespace}\medskip

        \begin{center}vorgelegt von\end{center}
        \begin{center}
            {\large Yahor Paromau} \\ geboren am 29.12.1998 in Hrodna
        \end{center}\medskip

        \begin{singlespace}\begin{center}\large
                Leibniz-Institut für Polymerforschung Dresden e. V. \\
                Fakultät Physik \\
                Bereich Mathematik und Naturwissenschaften \\
                Technische Universität Dresden \\ 2023
            \end{center}\end{singlespace}
    \end{tabularx}
\end{titlepage}


% Gutachterseite
\thispagestyle{empty}\vspace*{48em}

Eingereicht am xx.~Monat~20xx\vspace{1.5em}
\par{\large\begin{tabular}{ll}
        1. Gutachter: & Prof.~Dr.~XX \\
        2. Gutachter: & Prof.~Dr.~YY \\
    \end{tabular}}


% Abstractseite
\newpage
\begin{center}\large\bfseries Summary\end{center}


Abstract \\
\cite{Singh:2022} proposed, that binding of the GTPase Rab5 to the
long coiled-coil tethering protein EEA1 on an early endosome
induces a rigidity transition resulting in a large conformational
change in EEA1 from a rigid and extended to a flexible and collapsed state.


\vspace{20em}
Kurzzusammenfassung \\
Deutsch \\


% Inhaltsverzeichnis

\cleardoublepage

\thispagestyle{empty}
\tableofcontents

\cleardoublepage



\section{Introduction}
\subsection{Motivation}
\subsection{Polymer chain model}

In the realm of polymer physics, ideal chain models serve as crucial tools
for understanding the fundamental properties and behavior of polymer chains. 
These models provide simplified yet insightful descriptions of polymer chains' 
static properties and are integral to elucidating phenomena such as
polymer flexibility, persistence length, and overall chain behavior.
Ideal models are characterized by distinct assumptions concerning the 
permissible ranges of torsion and bond angles. Nevertheless, all 
these models disregard interactions between monomers that are widely 
spaced along the chain.
In this section, 3 relevant ideal chain models are described: 
the freely jointed chain model, freely rotating chain model and the worml-like chain model.

\subsubsection{Freely jointed chain model}
The freely jointed chain model is one of the simplest chain models. It assumes
a constant bond length $l$ and no correlations between the directions of different
bond vectors: $\E{cos(\theta_{ij})} = 0$ for $i \neq j$. This model is suitable
to study static properties of fully flexible polymer chains. In particular interest
are the end-to-end distance of the chain with $N$ bonds
\begin{equation}
    \E{R^2} = N l^2
\end{equation}
and the contour length 
\begin{equation}
    L = N l
\end{equation}

\subsubsection{Freely rotating chain model}

The freely rotating chain (FRC) model is a fundamental ideal chain model 
that offers a simple yet valuable perspective on polymer behavior. 
This model assumes that all bond lengths and all bond angles are fixed and
all torsion angles are equally likely and independent of each other, the bond angles can adopt any value without energetic 
penalty, akin to a series of interconnected rigid segments.
\\
The FRC model's implications are particularly noteworthy when 
studying the persistence length of polymers. 
This length scale characterizes the chain's ability to maintain a 
given direction before undergoing significant bending due to thermal motion. 


\subsubsection{Worm-like chain model}
The worm-like chain model is a special case of freely rotating chain model for
the small values of a bond angle: $\cos(\theta) << 1$ \cite{Rub_Colby_PolyPhy:2005}.
In this model the end-to-end distance of the chain can be written as:
\begin{equation}
    \E{R^2} = 2 l_p L - 2 l_p^2 (1-\exp(-\frac{L}{l_p}))
\end{equation}
And the following relation is usefull to estimate the persistence length of the chain:
\begin{equation}
    \E{cos(\theta_{ij})} = exp(-\frac{l |j-i|}{l_p})
\end{equation}

\subsection{Langevin equation}
The Langevin equation is a fundamental stochastic differential equation 
widely used in statistical mechanics and molecular dynamics to describe the 
dynamics of particles or molecules subjected to random forces in a 
dissipative medium.
\\
\\
Specifically Langevin equation is used to describe the particle in immobile
solvent and can be written as:
\begin{equation} \label{eq:langevin}
    m \ddot{\vec{r}} = - \nabla U(\vec{r}) - \sigma m \dot{\vec{r}} + \vec{f}^r(t)
\end{equation}
where:
\begin{itemize}
    \item $\sigma$ - damping constaint, which accounts for the viscosity of the solvent
    \item $\vec{f}^r(t)$ - stochastic force, represents the effect of thermal 
    fluctuations due to the particle's interactions with the solvents molecules
    \item $U(\vec{r})$ - any external potential acting on the particle
\end{itemize}
$\vec{f}^r(t)$ is a stochastic process known as white noise:
\begin{itemize}
    \item Sampled from Gaussian distribution
    \item $\E{\vec{f}^r(t)} = 0$
    \item $\E{\vec{f}_{\alpha}^r(t) \vec{f}_{\beta}^r(t')} = 2 k_B T \sigma m \delta(t-t')$ which relates strength
     of noise and friction and
    is known as fluctuation-dissipation theorem.
\end{itemize}

\subsection{Rouse model}
Polymer chains, composed of repeating monomer units, exhibit complex 
behaviors that are of great interest in polymer physics and materials science. 
Understanding the dynamics of polymer chains is crucial for elucidating their 
mechanical, thermal, and transport properties. One of the fundamental 
models used to describe polymer dynamics is the Rouse model.
\\
\\
The Rouse model, proposed by Peter J. W. Debye and Paul J. Flory in 
the 1940s, is a widely used theoretical framework to study the dynamics 
of polymer chains in an ideal solvent. This model simplifies the complex 
behavior of polymer chains by representing them as linear chains of connected beads.

\subsubsection{Assumptions}
The Rouse model makes certain key assumptions to facilitate its analytical treatment:
\begin{enumerate}
    \item \label{rouse_assumption_1} No hydrodynamic interactions or excluded volume effects between monomers.
    \item \label{rouse_assumption_2} Thermal forces acting on each bead follow Gaussian statistics.
    \item \label{rouse_assumption_3} Overdamped motion of the bead, which implies that inertia term vanishes: $m \ddot{\vec{r}} \approx 0$.
    Which is usually fulfilled in polymeric systems \cite{Doi_Intro_PP:2005}.
    \item \label{rouse_assumption_4} Beads continuously distributed along polymer chain.
\end{enumerate}

\subsubsection{Equation}
Assumptions \ref{rouse_assumption_1}, \ref{rouse_assumption_2} lead to description of the
system using Langevin equation (Eq. \ref{eq:langevin}). Following assumption \ref{rouse_assumption_3} 
the inertia term is set to 0. The continous approximation then made as consequence of assumption \ref{rouse_assumption_4}.
The initial equation of motion of single bead then becomes a diffusion equation \cite{Rub_Colby_PolyPhy:2005}:
\begin{equation}
    \label{eq:diffusion}
    \zeta \frac{\partial \vec{r}(t,n)}{\partial t} = \frac{3 k_B T}{l^2} \frac{\partial \vec{r}(t,n)}{\partial n^2} + \vec{f}^r(t)
\end{equation}

\subsubsection{Boundary conditions}
The chain ends are connected to one spring only. These free ends could be modelled by adding two 
fictious beads to both ends with $\vec{r}_0=\vec{r}_1$ and $\vec{r}_N=\vec{r}_{N+1}$. Boundary 
conditions for diffusion equation can be then written as follows \cite{Rub_Colby_PolyPhy:2005}:
\begin{equation}
    \label{eq:rouse_boundary}
    \begin{aligned}
        & \left(\frac{\partial \vec{r}}{\partial n}\right)_{n=0} = 0\\
        & \left(\frac{\partial \vec{r}}{\partial n}\right)_{n=N} = 0
    \end{aligned}
\end{equation}

\subsubsection{Solution}
The motion of the polymer can be decoupled in the motion of the independent modes using normal
coordinates \cite{Doi_Edwards_PD:1994}. Define:
\begin{equation}
    \label{eq:rouse_mode}
    \vec{X}_p := \frac{1}{N} \int_0^N dn \cos(\frac{p \pi n}{N})\vec{r}(n, t)
\end{equation}
Eq. \ref{eq:diffusion} can be then rewritten as \cite{Doi_Edwards_PD:1994}:
\begin{equation}
    \zeta_p \frac{\partial}{\partial t} \vec{X}_p = -k_p \vec{X}_p + \vec{f}^p
\end{equation}
with 
\begin{equation}
    \zeta_0 = N \zeta
    \text{ and }
    \zeta_p = 2 N \zeta
    \text{ for } p = 1,2,\ldots
\end{equation}
\begin{equation}
    k_p = \frac{6 \pi^2 k_B T}{Nl^2} p^2
    \text{ for } p = 0,1,2,\ldots
\end{equation}
and $\vec{f}^p$ is a random force, which satisfies \cite{Doi_Edwards_PD:1994}:
\begin{equation}
    \begin{aligned}
        & \E{f_{p\alpha}} = 0\\
        & \E{f_{p\alpha}(t) f_{p\beta}(t')} = 2 \delta_{pq} \delta_{\alpha\beta} \zeta_p k_B T \delta(t-t')
    \end{aligned}
\end{equation}
Which is langevin equation for the harmonic spring potential, with formal solution \cite{Doi_Edwards_PD:1994}:
\begin{equation}
    \vec{X}_p(t) = \frac{1}{\zeta_p} \int_{-\infty}^{t} dt' \frac{\exp(t-t')}{\tau_p} \vec{f^p}
\end{equation}
with
\begin{equation}
    \tau_p = \zeta_p / k_p
\end{equation}

$\vec{X}_p$ represents the local motion of the chain which includes $\frac{N}{p}$ segments and corresponds to the
motion with the length-scale of the order $\sqrt{Nl^2/p}$ \cite{Doi_Edwards_PD:1994}.
\\
\\
The inverse transformation of Eq. \ref{eq:rouse_mode} is given by \cite{Doi_Edwards_PD:1994}:
\begin{equation}
    \vec{r}_n = \vec{X_0} + 2 \sum_{p=1}^{\infty} \vec{X}_p \cos(p \pi n  / N)
\end{equation}

\subsubsection{Important values}
In this section the relevant dynamical quantities of rouse model are summarized.
\\
\\
Rouse relaxation time
\begin{equation}
    \tau_R := \tau_1 = \frac{1}{3 \pi^2} \frac{N^2 l^2 \zeta}{k_B T} 
\end{equation}
matches (except of factor $1/3\pi^2$) the time needed for the chain to diffuse over it's End-to-End distance.
\\
\\
Relaxation time of the single monomer
\begin{equation}
    \tau_0 = \frac{3 \pi^2 \tau_R}{N^2} = \frac{l^2 \zeta}{k_B T}
\end{equation}
matches the time needed for free particle of size $l$ to diffuse over distance
of it's own size, as described by Langevin equation for free particle.
\\
\\
Time correlation of End-to-End (ETE) vector \cite{Doi_Edwards_PD:1994}:
\begin{equation}
    \E{\vec{R}(t)\vec{R}(0)} = Nl^2 \sum_{p=1,3,\ldots} \frac{8}{p^2\pi^2}\exp(\frac{-t p^2}{\tau_R})
\end{equation}
\\
\\
Mean square displacement (MSD) of ETE
\begin{equation}
    \begin{aligned}
        \E{[\Delta \vec{R}(t)]^2} :&= \E{[\vec{R}(t)-\vec{R}(0)]^2} \\
        & = 2 N l^2 (1 - \frac{8}{\pi^2}\sum_{p=1,3,\ldots}\frac{1}{p^2} \exp(\frac{-tp^2}{\tau_R}))
    \end{aligned}
\end{equation}
is usefull for the description of the dynamics of the anchored chains end.
\\
\\
MSD of last monomer \cite{svaneborg_2020}:
\begin{equation}
    \E{[\Delta \vec{r}_N(t)]^2} = \frac{2}{\pi^2} \E{R^2} \left( \sum_{p=1}^{\infty} \frac{1}{p^2} \left[ 1-\exp(\frac{-t p^2}{\tau_R}) \right] + \frac{t}{\tau_R} \right)
\end{equation}
describes the dynamics of the chain end of free fully flexible chain.

\section{Methods}
\subsection{Simulation methods}
\subsubsection{Molecular dynamics}
Molecular dynamics (MD) is a powerful computational technique that
provides insight into the dynamic behavior of molecular systems by 
numerically solving the Newton's equations of motion for each atom or particle. 
In the context of polymer physics, MD simulations offer a detailed view of how polymer chains evolve over time, enabling the exploration of their conformational changes, interactions, and mechanical properties.
\\
\\
For the purpose of this study, the Large-scale Atomic/Molecular Massively
Parallel Simulator (LAMMPS) \cite{LAMMPS} is employed, a widely used software package 
for molecular dynamics simulations. LAMMPS facilitates the simulation setup and 
integration algorithms necessary to investigate the behavior of 
polymer chains with varying persistence length and boundary conditions.

\paragraph{Integrator}
In this study, the velocity-Verlet algorithm is employed as the numerical integrator. 
This algorithm updates particle positions and velocities over discrete time steps,
providing an accurate and stable trajectory for the simulation.
involves following steps \cite{10.1063/1.442716}:
\begin{enumerate}
    \item Calculate position: $\vec{r}(t+\Delta t) = \vec{r}(t) + \vec{v}(t) \Delta t + \frac{1}{2} \vec{a}(t) \Delta{t}^2$
    \item Derive $\vec{a}(t+\Delta t)$ from interaction potential using $\vec{r}(t+\Delta t)$
    \item Calculate $\vec{v}(t+\Delta t) = \vec{v} + \frac{1}{2}(\vec{a}(t) + \vec{a}(t+\Delta t))\Delta t$ 
\end{enumerate}

\paragraph{Potential}

\paragraph{Thermostat}


\subsubsection{Simulation setup}
\subsection{Evaluation methods}

\section{Results}
\subsection{Compare dynamics of fixed and free fully flexible chains}
\subsection{Dynamics of anchored chain at different stiffness values}
\subsection{Compare fixed rod-limit chain with and without large monomer tethered}
\subsection{Compare free rod-limit chain with and without large monomer tethered}

% Erklärung
\clearpage
\thispagestyle{empty}
\minisec{Erklärung}\vspace*{1.5em}

Hiermit erkläre ich, dass ich diese Arbeit im Rahmen der Betreuung am Institut
für ??? Physik ohne unzulässige Hilfe Dritter verfasst und alle Quellen als solche gekennzeichnet habe.

\vspace*{45em}

Vorname Nachname \par
Dresden, Monat 2019

\bibliographystyle{plain}
\bibliography{sources}

\end{document}
