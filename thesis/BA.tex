\documentclass[
    paper=A4,pagesize=automedia,fontsize=12pt,
    BCOR=15mm,DIV=22,
    twoside,headinclude,footinclude=false,
    ngerman,fleqn,             % fleqn = linksbündige Ausrichtung von Formeln
    bibliography=totocnumbered,          % Literaturverz. im Inhaltsverz. eintragen
    listof=totoc,                % Abbildungsverz. im Inhaltsverz. eintragen
    listof=flat,                 % Abbildungsverz. an der längsten Nummer ausrichten
    cleardoublepage=empty      % Vakatseiten ohne Paginierung
    numbers=endperiod
]{scrartcl}
\setlength\parindent{0em}

\usepackage{hyperref}
\hypersetup{
    colorlinks=false, %set true if you want colored links
    linktoc=all,     %set to all if you want both sections and subsections linked
    linkcolor=blue,  %choose some color if you want links to stand out
}

% Kodierung, Schrift und Sprache auswählen
\usepackage[utf8]{inputenc}
\usepackage[T1]{fontenc}
\usepackage[english]{babel}
% damit man Text aus dem PDF korrekt rauskopieren kann
\usepackage{cmap}
% Layout: Kopf-/Fußzeilen, anderthalbfacher Zeilenabstand
\usepackage[headsepline]{scrlayer-scrpage}
\pagestyle{scrheadings}
\clearpairofpagestyles
\ihead{\headmark}
\ohead{\pagemark}
\automark[subsection]{section}
\KOMAoption{headsepline}{0.5pt}

\usepackage{setspace}
\onehalfspacing
\deffootnote{1em}{1em}{\textsuperscript{\thefootnotemark}}

% Grafiken, Tabellen, Mathematikumgebungen
\usepackage{graphicx,xcolor}
\usepackage{tabularx}
\usepackage{amsmath,amsfonts,amssymb}

% Darstellung von Fließumgebungen
\usepackage{flafter,afterpage}
\usepackage[section]{placeins}
\usepackage[margin=8mm,font=small,labelfont=bf,format=plain]{caption}
\usepackage[margin=8mm,font=small,labelfont=bf,format=plain]{subcaption}

% \numberwithin{equation}{chapter}
% \numberwithin{figure}{chapter}
% \numberwithin{table}{chapter}

%%%%%%%%%%%%%%%%%%%%%%%%%%%%%%%%%%%%%%%%%%%%%%%%%%%%%%%%%%%%%%%%%%%%%%%%%%%%%%%%
% Ab hier ist Platz für eigene Ergänzungen (Pakete, Befehle, etc.)

% Dieses Paket liefert den Blindtext, der als Platzhalter in den Beispieldateien steht.
% Das kannst Du also entfernen, wenn Du den Blindtext nicht mehr brauchst.
\usepackage{lipsum}

\begin{document}

% \frontmatter


% Titelpageseite
\begin{titlepage}
    \begin{tabularx}{\linewidth}{X}
        \includegraphics[width=6cm]{img/TU_Logo_SW} \\\hline\hline

        \vspace{4.5em}

        \begin{singlespace}\begin{center}\bfseries\Huge

                Molecular dynamics study of ideal polymer chains
                for variable persistence length

            \end{center}\end{singlespace}

        \vspace{5.5em}

        \begin{singlespace}\begin{center}\large
                Bachelor-Arbeit \\ zur Erlangung des Hochschulgrades \\
                Bachelor of Science \\
                im Bachelor-Studiengang Physik
            \end{center}\end{singlespace}\medskip

        \begin{center}vorgelegt von\end{center}
        \begin{center}
            {\large Yahor Paromau} \\ geboren am 29.12.1998 in Hrodna
        \end{center}\medskip

        \begin{singlespace}\begin{center}\large
                Leibniz-Institut für Polymerforschung Dresden e. V. \\
                Fakultät Physik \\
                Bereich Mathematik und Naturwissenschaften \\
                Technische Universität Dresden \\ 2023
            \end{center}\end{singlespace}
    \end{tabularx}
\end{titlepage}


% Gutachterseite
\thispagestyle{empty}\vspace*{48em}

Eingereicht am xx.~Monat~20xx\vspace{1.5em}
\par{\large\begin{tabular}{ll}
        1. Gutachter: & Prof.~Dr.~XX \\
        2. Gutachter: & Prof.~Dr.~YY \\
    \end{tabular}}


% Abstractseite
\newpage
\begin{center}\large\bfseries Summary\end{center}


Abstract \\
\cite{Singh:2022} proposed, that binding of the GTPase Rab5 to the
long coiled-coil tethering protein EEA1 on an early endosome
induces a rigidity transition resulting in a large conformational
change in EEA1 from a rigid and extended to a flexible and collapsed state.


\vspace{20em}
Kurzzusammenfassung \\
Deutsch \\


% Inhaltsverzeichnis

\cleardoublepage

\thispagestyle{empty}
\tableofcontents

\cleardoublepage



\section{Introduction}
\subsection{Motivation}
\subsection{Polymer chain model}
\subsection{Rouse model of polymer dynamics}


\section{Methods}
\subsection{Simulation methods}
\subsection{Evaluation methods}

\section{Results}
\subsection{Compare dynamics of fixed and free fully flexible chains}
\subsection{Dynamics of anchored chain at different stiffness values}
\subsection{Compare fixed rod-limit chain with and without large monomer tethered}
\subsection{Compare free rod-limit chain with and without large monomer tethered}

% Erklärung
\clearpage
\thispagestyle{empty}
\minisec{Erklärung}\vspace*{1.5em}

Hiermit erkläre ich, dass ich diese Arbeit im Rahmen der Betreuung am Institut
für ??? Physik ohne unzulässige Hilfe Dritter verfasst und alle Quellen als solche gekennzeichnet habe.

\vspace*{45em}

Vorname Nachname \par
Dresden, Monat 2019

\bibliographystyle{plain}
\bibliography{sources}

\end{document}
