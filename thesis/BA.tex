\documentclass[
    paper=A4,pagesize=automedia,fontsize=12pt,
    BCOR=15mm,DIV=22,
    twoside,headinclude,footinclude=false,
    ngerman,fleqn,             % fleqn = linksbündige Ausrichtung von Formeln
    bibliography=totocnumbered,          % Literaturverz. im Inhaltsverz. eintragen
    listof=totoc,                % Abbildungsverz. im Inhaltsverz. eintragen
    listof=flat,                 % Abbildungsverz. an der längsten Nummer ausrichten
    cleardoublepage=empty      % Vakatseiten ohne Paginierung
    numbers=endperiod
]{scrartcl}
\setlength\parindent{0em}

\usepackage{hyperref}
\hypersetup{
    colorlinks=false, %set true if you want colored links
    linktoc=all,     %set to all if you want both sections and subsections linked
    linkcolor=blue,  %choose some color if you want links to stand out
}

% Kodierung, Schrift und Sprache auswählen
\usepackage[utf8]{inputenc}
\usepackage[T1]{fontenc}
\usepackage[english]{babel}
% damit man Text aus dem PDF korrekt rauskopieren kann
\usepackage{cmap}
% Layout: Kopf-/Fußzeilen, anderthalbfacher Zeilenabstand
\usepackage[headsepline]{scrlayer-scrpage}
\pagestyle{scrheadings}
\clearpairofpagestyles
\ihead{\headmark}
\ohead{\pagemark}
\automark[subsection]{section}
\KOMAoption{headsepline}{0.5pt}

\usepackage{setspace}
\onehalfspacing
\deffootnote{1em}{1em}{\textsuperscript{\thefootnotemark}}

% Grafiken, Tabellen, Mathematikumgebungen
\usepackage{graphicx,xcolor}
\usepackage{tabularx}
\usepackage{amsmath,amsfonts,amssymb}

% Darstellung von Fließumgebungen
\usepackage{flafter,afterpage}
\usepackage[section]{placeins}
\usepackage[margin=8mm,font=small,labelfont=bf,format=plain]{caption}
\usepackage[margin=8mm,font=small,labelfont=bf,format=plain]{subcaption}

% \numberwithin{equation}{chapter}
% \numberwithin{figure}{chapter}
% \numberwithin{table}{chapter}

%%%%%%%%%%%%%%%%%%%%%%%%%%%%%%%%%%%%%%%%%%%%%%%%%%%%%%%%%%%%%%%%%%%%%%%%%%%%%%%%
% Ab hier ist Platz für eigene Ergänzungen (Pakete, Befehle, etc.)

% Dieses Paket liefert den Blindtext, der als Platzhalter in den Beispieldateien steht.
% Das kannst Du also entfernen, wenn Du den Blindtext nicht mehr brauchst.
\usepackage{lipsum}

\begin{document}

% \frontmatter

% Commands
\newcommand{\E}[1]{\langle#1\rangle}

% Titelpageseite
\begin{titlepage}
    \begin{tabularx}{\linewidth}{X}
        \includegraphics[width=6cm]{img/TU_Logo_SW} \\\hline\hline

        \vspace{4.5em}

        \begin{singlespace}\begin{center}\bfseries\Huge

            Molecular dynamics study of ideal polymer chains with different persistence lengths

            \end{center}\end{singlespace}

        \vspace{5.5em}

        \begin{singlespace}\begin{center}\large
                Bachelor-Arbeit \\ zur Erlangung des Hochschulgrades \\
                Bachelor of Science \\
                im Bachelor-Studiengang Physik
            \end{center}\end{singlespace}\medskip

        \begin{center}vorgelegt von\end{center}
        \begin{center}
            {\large Yahor Paromau} \\ geboren am 29.12.1998 in Hrodna
        \end{center}\medskip

        \begin{singlespace}\begin{center}\large
                Leibniz-Institut für Polymerforschung Dresden e. V. \\
                Fakultät Physik \\
                Bereich Mathematik und Naturwissenschaften \\
                Technische Universität Dresden \\ 2023
            \end{center}\end{singlespace}
    \end{tabularx}
\end{titlepage}


% Gutachterseite
\thispagestyle{empty}\vspace*{48em}

Eingereicht am xx.~Monat~20xx\vspace{1.5em}
\par{\large\begin{tabular}{ll}
        1. Gutachter: & Prof.~Dr.~XX \\
        2. Gutachter: & Prof.~Dr.~YY \\
    \end{tabular}}


% Abstractseite
\newpage
\begin{center}\large\bfseries Summary\end{center}


Abstract \\
\cite{Singh:2022} proposed, that binding of the GTPase Rab5 to the
long coiled-coil tethering protein EEA1 on an early endosome
induces a rigidity transition resulting in a large conformational
change in EEA1 from a rigid and extended to a flexible and collapsed state.


\vspace{20em}
Kurzzusammenfassung \\
Deutsch \\


% Inhaltsverzeichnis

\cleardoublepage

\thispagestyle{empty}
\tableofcontents

\cleardoublepage



\section{Introduction}
\subsection{Motivation}
\subsection{Polymer chain model}
\subsection{Langevin equation}
The Langevin equation is a fundamental stochastic differential equation 
widely used in statistical mechanics and molecular dynamics to describe the 
dynamics of particles or molecules subjected to random forces in a 
dissipative medium.
\\
\\
Specifically Langevin equation is used to describe the particle in immobile
solvent and can be written as:
\begin{equation} \label{eq:langevin}
    m \ddot{\vec{r}} = - \nabla U(\vec{r}) - \sigma m \dot{\vec{r}} + \vec{f}^r(t)
\end{equation}
where:
\begin{itemize}
    \item $\sigma$ - damping constaint, which accounts for the viscosity of the solvent
    \item $\vec{f}^r(t)$ - stochastic force, represents the effect of thermal 
    fluctuations due to the particle's interactions with the solvents molecules
    \item $U(\vec{r})$ - any external potential acting on the particle
\end{itemize}
$\vec{f}^r(t)$ is a stochastic process known as white noise:
\begin{itemize}
    \item Sampled from Gaussian distribution
    \item $\E{\vec{f}^r(t)} = 0$
    \item $\E{\vec{f}_{\alpha}^r(t) \vec{f}_{\beta}^r(t')} = 2 k_B T \sigma m \delta(t-t')$ which relates strength
     of noise and friction and
    is known as fluctuation-dissipation theorem.
\end{itemize}

\subsection{Rouse model}
Polymer chains, composed of repeating monomer units, exhibit complex 
behaviors that are of great interest in polymer physics and materials science. 
Understanding the dynamics of polymer chains is crucial for elucidating their 
mechanical, thermal, and transport properties. One of the fundamental 
models used to describe polymer dynamics is the Rouse model.
\\
\\
The Rouse model, proposed by Peter J. W. Debye and Paul J. Flory in 
the 1940s, is a widely used theoretical framework to study the dynamics 
of polymer chains in an ideal solvent. This model simplifies the complex 
behavior of polymer chains by representing them as linear chains of connected beads.

\subsubsection{Assumptions}
The Rouse model makes certain key assumptions to facilitate its analytical treatment:
\begin{enumerate}
    \item \label{rouse_assumption_1} No hydrodynamic interactions or excluded volume effects between monomers.
    \item \label{rouse_assumption_2} Thermal forces acting on each bead follow Gaussian statistics.
    \item \label{rouse_assumption_3} Overdamped motion of the bead, which implies that inertia term vanishes: $m \ddot{\vec{r}} \approx 0$.
    Which is usually fulfilled in polymeric systems \cite{Doi_Intro_PP:2005}.
    \item \label{rouse_assumption_4} Beads continuously distributed along polymer chain.
\end{enumerate}

\subsubsection{Equation}
Assumptions \ref{rouse_assumption_1}, \ref{rouse_assumption_2} lead to description of the
system using Langevin equation (Eq. \ref{eq:langevin}). Following assumption \ref{rouse_assumption_3} 
the inertia term is set to 0. The continous approximation then made as consequence of assumption \ref{rouse_assumption_4}.
The initial equation of motion of single bead then becomes a diffusion equation \cite{Rub_Colby_PolyPhy:2005}:
\begin{equation}
    \label{eq:diffusion}
    \zeta \frac{\partial \vec{r}(t,n)}{\partial t} = \frac{3 k_B T}{b^2} \frac{\partial \vec{r}(t,n)}{\partial n^2} + \vec{f}^r(t)
\end{equation}

\subsubsection{Boundary conditions}
The chain ends are connected to one spring only. These free ends could be modelled by adding two 
fictious beads to both ends with $\vec{r}_0=\vec{r}_1$ and $\vec{r}_N=\vec{r}_{N+1}$. Boundary 
conditions for diffusion equation can be then written as follows \cite{Rub_Colby_PolyPhy:2005}:
\begin{equation}
    \label{eq:rouse_boundary}
    \begin{aligned}
        & \left(\frac{\partial \vec{r}}{\partial n}\right)_{n=0} = 0\\
        & \left(\frac{\partial \vec{r}}{\partial n}\right)_{n=N} = 0
    \end{aligned}
\end{equation}

\subsubsection{Solution}
Define:
\begin{equation}
    \label{eq:rouse_mode}
    \vec{X}_p := \frac{1}{N} \int_0^N dn \cos(\frac{p \pi n}{N})\vec{r}(n, t)
\end{equation}
Eq. \ref{eq:diffusion} can be then rewritten as \cite{Doi_Edwards_PD:1994}:
\begin{equation}
    \zeta_p \frac{\partial}{\partial t} \vec{X}_p = -k_p \vec{X}_p + \vec{f}^p
\end{equation}
with 
\begin{equation}
    \zeta_0 = N \zeta
    \text{ and }
    \zeta_p = 2 N \zeta
    \text{ for } p = 1,2,\ldots
\end{equation}
\begin{equation}
    k_p = \frac{6 \pi^2 k_B T}{Nb^2} p^2
    \text{ for } p = 0,1,2,\ldots
\end{equation}
and $\vec{f}^p$ is a random force, which satisfies \cite{Doi_Edwards_PD:1994}:
\begin{equation}
    \begin{aligned}
        & \E{f_{p\alpha}} = 0\\
        & \E{f_{p\alpha}(t) f_{p\beta}(t')} = 2 \delta_{pq} \delta_{\alpha\beta} \zeta_p k_B T \delta(t-t')
    \end{aligned}
\end{equation}
Which is langevin equation
\section{Methods}
\subsection{Simulation methods}
\subsection{Evaluation methods}

\section{Results}
\subsection{Compare dynamics of fixed and free fully flexible chains}
\subsection{Dynamics of anchored chain at different stiffness values}
\subsection{Compare fixed rod-limit chain with and without large monomer tethered}
\subsection{Compare free rod-limit chain with and without large monomer tethered}

% Erklärung
\clearpage
\thispagestyle{empty}
\minisec{Erklärung}\vspace*{1.5em}

Hiermit erkläre ich, dass ich diese Arbeit im Rahmen der Betreuung am Institut
für ??? Physik ohne unzulässige Hilfe Dritter verfasst und alle Quellen als solche gekennzeichnet habe.

\vspace*{45em}

Vorname Nachname \par
Dresden, Monat 2019

\bibliographystyle{plain}
\bibliography{sources}

\end{document}
