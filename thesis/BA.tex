\input{preamble}
\begin{document}

\frontmatter


% Titelpageseite
\begin{titlepage}
    \begin{tabularx}{\linewidth}{X}
        \includegraphics[width=6cm]{img/TU_Logo_SW} \\\hline\hline

        \vspace{4.5em}

        \begin{singlespace}\begin{center}\bfseries\Huge

                Molecular dynamics study of ideal polymer chains
                for variable persistence length

            \end{center}\end{singlespace}

        \vspace{5.5em}

        \begin{singlespace}\begin{center}\large
                Bachelor-Arbeit \\ zur Erlangung des Hochschulgrades \\
                Bachelor of Science \\
                im Bachelor-Studiengang Physik
            \end{center}\end{singlespace}\medskip

        \begin{center}vorgelegt von\end{center}
        \begin{center}
            {\large Yahor Paromau} \\ geboren am 29.12.1998 in Hrodna
        \end{center}\medskip

        \begin{singlespace}\begin{center}\large
                Leibniz-Institut für Polymerforschung Dresden e. V. \\
                Fakultät Physik \\
                Bereich Mathematik und Naturwissenschaften \\
                Technische Universität Dresden \\ 2023
            \end{center}\end{singlespace}
    \end{tabularx}
\end{titlepage}


% Gutachterseite
\thispagestyle{empty}\vspace*{48em}

Eingereicht am xx.~Monat~20xx\vspace{1.5em}
\par{\large\begin{tabular}{ll}
        1. Gutachter: & Prof.~Dr.~XX \\
        2. Gutachter: & Prof.~Dr.~YY \\
    \end{tabular}}


% Abstractseite
\newpage
\begin{center}\large\bfseries Summary\end{center}


Abstract \\
\cite{Singh:2022} proposed, that binding of the GTPase Rab5 to the
long coiled-coil tethering protein EEA1 on an early endosome
induces a rigidity transition resulting in a large conformational
change in EEA1 from a rigid and extended to a flexible and collapsed state.


\vspace{20em}
Kurzzusammenfassung \\
Deutsch \\


% Inhaltsverzeichnis

%\cleardoublepage
\tableofcontents



% Hauptteil

\chapter{Introduction}

\chapter{Main Part}

\chapter{Summary and Outlook}


% Erklärung
\clearpage
\thispagestyle{empty}
\minisec{Erklärung}\vspace*{1.5em}

Hiermit erkläre ich, dass ich diese Arbeit im Rahmen der Betreuung am Institut
für ??? Physik ohne unzulässige Hilfe Dritter verfasst und alle Quellen als solche gekennzeichnet habe.

\vspace*{45em}

Vorname Nachname \par
Dresden, Monat 2019

\bibliographystyle{plain}
\bibliography{sources}

\end{document}
