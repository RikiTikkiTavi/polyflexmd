\documentclass[
    paper=A4,pagesize=automedia,fontsize=12pt,
    BCOR=15mm,DIV=22,
    twoside,headinclude,footinclude=false,
    ngerman,fleqn,             % fleqn = linksbündige Ausrichtung von Formeln
    bibliography=totocnumbered,          % Literaturverz. im Inhaltsverz. eintragen
    listof=totoc,                % Abbildungsverz. im Inhaltsverz. eintragen
    listof=flat,                 % Abbildungsverz. an der längsten Nummer ausrichten
    cleardoublepage=empty      % Vakatseiten ohne Paginierung
    numbers=endperiod
]{scrartcl}
\setlength\parindent{0em}

\usepackage{hyperref}
\hypersetup{
    colorlinks=false, %set true if you want colored links
    linktoc=all,     %set to all if you want both sections and subsections linked
    linkcolor=blue,  %choose some color if you want links to stand out
}

% Kodierung, Schrift und Sprache auswählen
\usepackage[utf8]{inputenc}
\usepackage[T1]{fontenc}
\usepackage[english]{babel}
% damit man Text aus dem PDF korrekt rauskopieren kann
\usepackage{cmap}
% Layout: Kopf-/Fußzeilen, anderthalbfacher Zeilenabstand
\usepackage[headsepline]{scrlayer-scrpage}
\pagestyle{scrheadings}
\clearpairofpagestyles
\ihead{\headmark}
\ohead{\pagemark}
\automark[subsection]{section}
\KOMAoption{headsepline}{0.5pt}

\usepackage{setspace}
\onehalfspacing
\deffootnote{1em}{1em}{\textsuperscript{\thefootnotemark}}

% Grafiken, Tabellen, Mathematikumgebungen
\usepackage{graphicx,xcolor}
\usepackage{tabularx}
\usepackage{amsmath,amsfonts,amssymb}

% Darstellung von Fließumgebungen
\usepackage{flafter,afterpage}
\usepackage[section]{placeins}
\usepackage[margin=8mm,font=small,labelfont=bf,format=plain]{caption}
\usepackage[margin=8mm,font=small,labelfont=bf,format=plain]{subcaption}

% \numberwithin{equation}{chapter}
% \numberwithin{figure}{chapter}
% \numberwithin{table}{chapter}

%%%%%%%%%%%%%%%%%%%%%%%%%%%%%%%%%%%%%%%%%%%%%%%%%%%%%%%%%%%%%%%%%%%%%%%%%%%%%%%%
% Ab hier ist Platz für eigene Ergänzungen (Pakete, Befehle, etc.)

% Dieses Paket liefert den Blindtext, der als Platzhalter in den Beispieldateien steht.
% Das kannst Du also entfernen, wenn Du den Blindtext nicht mehr brauchst.
\usepackage{lipsum}

\usepackage{booktabs}

\graphicspath{ {./img/} }

\begin{document}

% \frontmatter

% Commands
\newcommand{\E}[1]{\langle#1\rangle}

% Titelpageseite
\begin{titlepage}
    \begin{tabularx}{\linewidth}{X}
        \includegraphics[width=6cm]{TU_Logo_SW} \\\hline\hline

        \vspace{4.5em}

        \begin{singlespace}\begin{center}\bfseries\Huge

            Molecular dynamics study of ideal polymer chains with different persistence lengths

            \end{center}\end{singlespace}

        \vspace{5.5em}

        \begin{singlespace}\begin{center}\large
                Bachelor-Arbeit \\ zur Erlangung des Hochschulgrades \\
                Bachelor of Science \\
                im Bachelor-Studiengang Physik
            \end{center}\end{singlespace}\medskip

        \begin{center}vorgelegt von\end{center}
        \begin{center}
            {\large Yahor Paromau} \\ geboren am 29.12.1998 in Hrodna
        \end{center}\medskip

        \begin{singlespace}\begin{center}\large
                Leibniz-Institut für Polymerforschung Dresden e. V. \\
                Fakultät Physik \\
                Bereich Mathematik und Naturwissenschaften \\
                Technische Universität Dresden \\ 2023
            \end{center}\end{singlespace}
    \end{tabularx}
\end{titlepage}


% Gutachterseite
\thispagestyle{empty}\vspace*{48em}

Eingereicht am xx.~Monat~20xx\vspace{1.5em}
\par{\large\begin{tabular}{ll}
        1. Gutachter: & Prof.~Dr.~XX \\
        2. Gutachter: & Prof.~Dr.~YY \\
    \end{tabular}}


% Abstractseite
\newpage
\begin{center}\large\bfseries Summary\end{center}


Abstract \\
\cite{Singh:2022} proposed, that binding of the GTPase Rab5 to the
long coiled-coil tethering protein EEA1 on an early endosome
induces a rigidity transition resulting in a large conformational
change in EEA1 from a rigid and extended to a flexible and collapsed state.


\vspace{20em}
Kurzzusammenfassung \\
Deutsch \\


% Inhaltsverzeichnis

\cleardoublepage

\thispagestyle{empty}
\tableofcontents

\cleardoublepage



\section{Introduction}
\subsection{Motivation}
\subsection{Polymer chain model}

In the realm of polymer physics, ideal chain models serve as crucial tools
for understanding the fundamental properties and behavior of polymer chains. 
These models provide simplified yet insightful descriptions of polymer chains' 
static properties and are integral to elucidating phenomena such as
polymer flexibility, persistence length, and overall chain behavior.
Ideal models are characterized by distinct assumptions concerning the 
permissible ranges of torsion and bond angles. Nevertheless, all 
these models disregard interactions between monomers that are widely 
spaced along the chain.
In this section, 3 relevant ideal chain models are described: 
the freely jointed chain model, freely rotating chain model and the worml-like chain model.

\subsubsection{Freely jointed chain model}
The freely jointed chain model is one of the simplest chain models. It assumes
a constant bond length $l$ and no correlations between the directions of different
bond vectors: $\E{cos(\theta_{ij})} = 0$ for $i \neq j$. This model is suitable
to study static properties of fully flexible polymer chains. In particular interest
are the end-to-end distance of the chain with $N$ bonds
\begin{equation}
    \E{R^2} = N l^2
\end{equation}
and the contour length 
\begin{equation}
    L = N l
\end{equation}

\subsubsection{Freely rotating chain model}

The freely rotating chain (FRC) model is a fundamental ideal chain model 
that offers a simple yet valuable perspective on polymer behavior. 
This model assumes that all bond lengths and all bond angles are fixed and
all torsion angles are equally likely and independent of each other, the bond angles can adopt any value without energetic 
penalty, akin to a series of interconnected rigid segments.
\\
The FRC model's implications are particularly noteworthy when 
studying the persistence length of polymers. 
This length scale characterizes the chain's ability to maintain a 
given direction before undergoing significant bending due to thermal motion. 


\subsubsection{Worm-like chain model}
The worm-like chain model is a special case of freely rotating chain model for
the small values of a bond angle: $\cos(\theta) << 1$ \cite{Rub_Colby_PolyPhy:2005}.
In this model the end-to-end distance of the chain can be written as:
\begin{equation}
    \E{R^2} = 2 l_p L - 2 l_p^2 (1-\exp(-\frac{L}{l_p}))
\end{equation}
And the following relation is usefull to estimate the persistence length of the chain:
\begin{equation}
    \E{cos(\theta_{ij})} = exp(-\frac{l |j-i|}{l_p})
\end{equation}

\subsection{Langevin equation} \label{section:langevin_eq}
The Langevin equation is a fundamental stochastic differential equation 
widely used in statistical mechanics and molecular dynamics to describe the 
dynamics of particles or molecules subjected to random forces in a 
dissipative medium.
\\
\\
Specifically Langevin equation is used to describe the particle in immobile
solvent and can be written as:
\begin{equation} \label{eq:langevin}
    m \ddot{\vec{r}} = - \nabla U(\vec{r}) - \sigma m \dot{\vec{r}} + \vec{f}^r(t)
\end{equation}
where:
\begin{itemize}
    \item $\sigma$ - damping constaint, which accounts for the viscosity of the solvent
    \item $\vec{f}^r(t)$ - stochastic force, represents the effect of thermal 
    fluctuations due to the particle's interactions with the solvents molecules
    \item $U(\vec{r})$ - any external potential acting on the particle
\end{itemize}
$\vec{f}^r(t)$ is a stochastic process known as white noise:
\begin{itemize}
    \item Sampled from Gaussian distribution
    \item $\E{\vec{f}^r(t)} = 0$
    \item $\E{\vec{f}_{\alpha}^r(t) \vec{f}_{\beta}^r(t')} = 2 k_B T \sigma m \delta(t-t')$ which relates strength
     of noise and friction and
    is known as fluctuation-dissipation theorem.
\end{itemize}

\subsection{Rouse model}
Polymer chains, composed of repeating monomer units, exhibit complex 
behaviors that are of great interest in polymer physics and materials science. 
Understanding the dynamics of polymer chains is crucial for elucidating their 
mechanical, thermal, and transport properties. One of the fundamental 
models used to describe polymer dynamics is the Rouse model.
\\
\\
The Rouse model is a widely used theoretical framework to study the dynamics 
of polymer chains in an ideal solvent. This model simplifies the complex 
behavior of polymer chains by representing them as linear chains of connected beads.

\subsubsection{Assumptions}
The Rouse model makes certain key assumptions to facilitate its analytical treatment:
\begin{enumerate}
    \item \label{rouse_assumption_1} No hydrodynamic interactions or excluded volume effects between monomers.
    \item \label{rouse_assumption_2} Thermal forces acting on each bead follow Gaussian statistics.
    \item \label{rouse_assumption_3} Overdamped motion of the bead, which implies that inertia term vanishes: $m \ddot{\vec{r}} \approx 0$.
    Which is usually fulfilled in polymeric systems \cite{Doi_Intro_PP:2005}.
    \item \label{rouse_assumption_4} Beads continuously distributed along polymer chain.
\end{enumerate}

\subsubsection{Equation}
Assumptions \ref{rouse_assumption_1}, \ref{rouse_assumption_2} lead to description of the
system using Langevin equation (Eq. \ref{eq:langevin}). Following assumption \ref{rouse_assumption_3} 
the inertia term is set to 0. The continous approximation then made as consequence of assumption \ref{rouse_assumption_4}.
The initial equation of motion of single bead then becomes a diffusion equation \cite{Rub_Colby_PolyPhy:2005}:
\begin{equation}
    \label{eq:diffusion}
    \zeta \frac{\partial \vec{r}(t,n)}{\partial t} = \frac{3 k_B T}{l^2} \frac{\partial \vec{r}(t,n)}{\partial n^2} + \vec{f}^r(t)
\end{equation}

\subsubsection{Boundary conditions}
The chain ends are connected to one spring only. These free ends could be modelled by adding two 
fictious beads to both ends with $\vec{r}_0=\vec{r}_1$ and $\vec{r}_N=\vec{r}_{N+1}$. Boundary 
conditions for diffusion equation can be then written as follows \cite{Rub_Colby_PolyPhy:2005}:
\begin{equation}
    \label{eq:rouse_boundary}
    \begin{aligned}
        & \left(\frac{\partial \vec{r}}{\partial n}\right)_{n=0} = 0\\
        & \left(\frac{\partial \vec{r}}{\partial n}\right)_{n=N} = 0
    \end{aligned}
\end{equation}

\subsubsection{Solution}
The motion of the polymer can be decoupled in the motion of the independent modes using normal
coordinates \cite{Doi_Edwards_PD:1994}. Define:
\begin{equation}
    \label{eq:rouse_mode}
    \vec{X}_p := \frac{1}{N} \int_0^N dn \cos(\frac{p \pi n}{N})\vec{r}(n, t)
\end{equation}
Eq. \ref{eq:diffusion} can be then rewritten as \cite{Doi_Edwards_PD:1994}:
\begin{equation}
    \zeta_p \frac{\partial}{\partial t} \vec{X}_p = -k_p \vec{X}_p + \vec{f}^p
\end{equation}
with 
\begin{equation}
    \zeta_0 = N \zeta
    \text{ and }
    \zeta_p = 2 N \zeta
    \text{ for } p = 1,2,\ldots
\end{equation}
\begin{equation}
    k_p = \frac{6 \pi^2 k_B T}{Nl^2} p^2
    \text{ for } p = 0,1,2,\ldots
\end{equation}
and $\vec{f}^p$ is a random force, which satisfies \cite{Doi_Edwards_PD:1994}:
\begin{equation}
    \begin{aligned}
        & \E{f_{p\alpha}} = 0\\
        & \E{f_{p\alpha}(t) f_{p\beta}(t')} = 2 \delta_{pq} \delta_{\alpha\beta} \zeta_p k_B T \delta(t-t')
    \end{aligned}
\end{equation}
Which is langevin equation for the harmonic spring potential, with formal solution \cite{Doi_Edwards_PD:1994}:
\begin{equation}
    \vec{X}_p(t) = \frac{1}{\zeta_p} \int_{-\infty}^{t} dt' \frac{\exp(t-t')}{\tau_p} \vec{f^p}
\end{equation}
with
\begin{equation}
    \tau_p = \zeta_p / k_p
\end{equation}

$\vec{X}_p$ represents the local motion of the chain which includes $\frac{N}{p}$ segments and corresponds to the
motion with the length-scale of the order $\sqrt{Nl^2/p}$ \cite{Doi_Edwards_PD:1994}.
\\
\\
The inverse transformation of Eq. \ref{eq:rouse_mode} is given by \cite{Doi_Edwards_PD:1994}:
\begin{equation}
    \vec{r}_n = \vec{X_0} + 2 \sum_{p=1}^{\infty} \vec{X}_p \cos(p \pi n  / N)
\end{equation}

\subsubsection{Important values}
In this section the relevant dynamical quantities of rouse model are summarized.
\\
\\
Rouse relaxation time
\begin{equation} \label{eq:rouse_relaxation_time}
    \tau_R := \tau_1 = \frac{1}{3 \pi^2} \frac{N^2 l^2 \zeta}{k_B T} 
\end{equation}
matches (except of factor $1/3\pi^2$) the time needed for the chain to diffuse over it's End-to-End distance.
\\
\\
Relaxation time of the single monomer
\begin{equation}
    \tau_0 = \frac{3 \pi^2 \tau_R}{N^2} = \frac{l^2 \zeta}{k_B T}
\end{equation}
matches the time needed for free particle of size $l$ to diffuse over distance
of it's own size, as described by Langevin equation for free particle.
\\
\\
Time correlation of End-to-End (ETE) vector \cite{Doi_Edwards_PD:1994}:
\begin{equation}
    \E{\vec{R}(t)\vec{R}(0)} = Nl^2 \sum_{p=1,3,\ldots} \frac{8}{p^2\pi^2}\exp(\frac{-t p^2}{\tau_R})
\end{equation}
\\
\\
Mean square displacement (MSD) of ETE
\begin{equation} \label{eq:rouse_msd_ete}
    \begin{aligned}
        \E{[\Delta \vec{R}(t)]^2} :&= \E{[\vec{R}(t)-\vec{R}(0)]^2} \\
        & = 2 N l^2 (1 - \frac{8}{\pi^2}\sum_{p=1,3,\ldots}\frac{1}{p^2} \exp(\frac{-tp^2}{\tau_R}))
    \end{aligned}
\end{equation}
is usefull for the description of the dynamics of the anchored chains end.
\\
\\
MSD of last monomer \cite{svaneborg_2020}:
\begin{equation}
    \E{[\Delta \vec{r}_N(t)]^2} = \frac{2}{\pi^2} \E{R^2} \left( \sum_{p=1}^{\infty} \frac{1}{p^2} \left[ 1-\exp(\frac{-t p^2}{\tau_R}) \right] + \frac{t}{\tau_R} \right)
\end{equation}
describes the dynamics of the chain end of free fully flexible chain.

\section{Methods}
\subsection{Simulation methods}
\subsubsection{Molecular dynamics}
Molecular dynamics (MD) is a powerful computational technique that
provides insight into the dynamic behavior of molecular systems by 
numerically solving the Newton's equations of motion for each atom or particle. 
In the context of polymer physics, MD simulations offer a detailed view of how polymer chains evolve over time, enabling the exploration of their conformational changes, interactions, and mechanical properties.
\\
\\
For the purpose of this study, the Large-scale Atomic/Molecular Massively
Parallel Simulator (LAMMPS) \cite{LAMMPS} is employed, a widely used software package 
for molecular dynamics simulations. LAMMPS facilitates the simulation setup and 
integration algorithms necessary to investigate the behavior of 
polymer chains with varying persistence length and boundary conditions.

\paragraph{Integrator}
In this study, the velocity-Verlet algorithm is employed as the numerical integrator. 
This algorithm updates particle positions and velocities over discrete time steps,
providing an accurate and stable trajectory for the simulation.
Involves following steps \cite{10.1063/1.442716}:
\begin{enumerate}
    \item Calculate position: $\vec{r}(t+\Delta t) = \vec{r}(t) + \vec{v}(t) \Delta t + \frac{1}{2} \vec{a}(t) \Delta{t}^2$
    \item Derive $\vec{a}(t+\Delta t)$ from interaction potential using $\vec{r}(t+\Delta t)$
    \item Calculate $\vec{v}(t+\Delta t) = \vec{v} + \frac{1}{2}(\vec{a}(t) + \vec{a}(t+\Delta t))\Delta t$ 
\end{enumerate}

\paragraph{Bond potential}
\label{par:bond_potential}
The finite extensible nonlinear elastic (FENE) potential \cite{Kremer_ChemPhys} 
is used to model the bonds between neighbour-monomers. This is a standard 
choice for bead-spring polymer models \cite{LAMMPS}. The potential equation used in
LAMMPS can be written as \cite{LAMMPS}:

\begin{equation}
    U_{bond}(r) = 
    \begin{cases}
        -\frac{1}{2} k R_0^2 \ln\left[1 - \left(\frac{r}{R_0}\right)^2\right] + 4 \epsilon \left[\left(\frac{\sigma}{r}\right)^{12} - \left(\frac{\sigma}{r}\right)^6\right] + \epsilon & \text{if } r <= 2^{1/6} \sigma \\
        -\frac{1}{2} k R_0^2 \ln\left[1 - \left(\frac{r}{R_0}\right)^2\right] & \text{if } 2^{1/6}\sigma < r <= R_0\\
        0 & \text{else}
    \end{cases}
\end{equation}

The potential consists of attractive (first) and repulsive (Lennard-Jones, second) term which is set to $0$ if $r$ is larger then
the minimum ($2^{1/6}\sigma$) of Lennard-Jones potential.

\paragraph{Bending potential}
\label{par:bend_potential}
An entropic worm-like potential is introduced to control the persistence length of the chain \cite{svaneborg_2020}.
The potential is defined by:
\begin{equation}
    U_{bend}(\theta) = \kappa (1 - \cos(\theta))
\end{equation}

\paragraph{Thermostat}
Langevin thermostat is used to regulate the temperature of the system and interaction with the solvent.
The Langevin thermostat emulates the interactions between particles and a heat bath, 
introducing stochastic forces to individual particles that mimic the effects of thermal fluctuations,
as well as interactions between particles and solvent introducing friction force to individual particles.
The equations of motion for individual particle then take a form of Langevin equation as described
in section \ref{section:langevin_eq} aside from some implementation-driven properties of random force $\vec{f}^r$:
\begin{enumerate}
    \item $\vec{f}^r \propto \sqrt{k_B T m \sigma \frac{1}{dt}}$ where $dt$ is integration timestep \cite{LAMMPS}
    \item The uniform random number is used to to randomize the direction 
    and magnitude of this force instead of Gaussian random number to speedup the
    calculations \cite{LAMMPS} \cite{dunweg}.
\end{enumerate}

\paragraph{Boundaries}
Periodic boundary conditions (PBC) are introduced to mimic the behavior of the infinite system.
This approach eliminates boundary effects and creates a virtual environment where particles 
interact as if they were part of a continuous and unbounded space.
Under PBC, when a particle exits one edge of the simulation box, it re-enters from the opposite edge, 
maintaining the illusion of a seamless and infinite system. This study includes only the interaction
between bonded monomers and adjacent bonds so the problem of "image" interactions is excluded. 


\subsubsection{Simulation setup}
Chosen simulation settings and parameters are explained in this section.
\\
\\
LJ unit system was set for the simulation. In this system 
mass, $\sigma, \epsilon, k_B = 1$ and the masses, distances, energies are specified
as multiples of this values \cite{LAMMPS}.
\\
\\
Simulation box has side length $200$ and periodic boundary conditions in all 
dimensions are used. The polymer chain is modelled as series of interconnected
beads, where bead represent a monomer and bonds represent the bonds between monomers.
The adjucent connected beads of the chain interract according to FENE potential $U_{bond}(r)$
as introduced in section \ref{par:bond_potential} with parameters: 
$K=30.0$, $R_0=1.5$, $\epsilon=1.0$, $\sigma=1.0$.
The angles between adjacent bonds of the same chain are affected by bending potential
$U_{bend}(\theta)$ as described in section \ref{par:bend_potential}, $\kappa$ is varied to obtain
desired Kuhn length according to equation \cite{svaneborg_2020}:
\begin{equation}
    l_K = l_b 
    \begin{cases}
        \frac{2 \kappa + \exp(-2 \kappa) - 1}{1-\exp(-2\kappa)(2\kappa + 1)} & \text{if } \kappa > 0 \\
        1 & \text{if } \kappa = 0 
    \end{cases}
\end{equation}
\\
\\
The simulation box is populated with 500 chains. Each chain consists of 64 monomers of mass $m=1$
and initial bond length $l_b=0.97$ with start in $0$ and is created using random walk. 
The chains do not interract between each other and hence form
an ensemble of 500 chains. In case of experiments with larger chain end the mass of the end monomer
is set $m_e=1.5$. In case of experiments with large values of Kuhn length the angles are randomized in 
the way to have more straight chain to resulting in decreasing number of steps to achive the
equilibrium.
\\
\\
The temperature of the system and forces on the beads are controlled with
Langevin thermostat with parameters: $\text{damp}=\frac{1}{\sigma}=1.0$, $T=1.0$. In case of anchored
chain the forces acting on first two beads of the chain are set to 0 pinning the chain to the origin
and excluding rotational degrees of freedom.
\\
\\
The simulation is runned until it reaches the equilibrium and then the measurements
of particle trajectories are performed. 

\subsection{Evaluation methods}
\paragraph{Main-axis coordinate system}
To analyze the dynamics of anchored chain the so called main-axis coordinate
system is introduced. In this system the $z$ axis is parallel to the
first bond vector, connecting the first two beads, which are fixed.
The orthonormal basis set that defines the system is calculated as follows:
\begin{equation}
    \hat{z}_{MA} = \frac{\vec{r}_1 - \vec{r}_0}{\| \vec{r}_1 - \vec{r}_0 \|}
\end{equation}
\begin{equation}
    \hat{y}_{MA} = 
    \begin{cases}
        (\hat{x} + \hat{y}) \times \hat{z}_{MA} & \text{if } \hat{x} \parallel \hat{z}_{MA} \\
        \hat{x} \times \hat{z}_{MA} & \text{otherwise}
    \end{cases}
\end{equation}
\begin{equation}
    \hat{x}_{MA} = \hat{z}_{MA} \times \hat{y}_{MA} 
\end{equation}

\section{Results}
This section unveils the outcomes of an investigation into the dynamic
characteristics of anchored and free polymer chains. 
The focus of the study lies in assessing the impact 
of two properties: chain stiffness and the friction of chain end.

\subsection{Anchored chain dynamics}
Firstly, this study focuses on a a comparative analysis between the 
dynamics of anchored polymer chains and their free counterparts.
To execute this comparison, the focus is narrowed onto a specific dynamical
quantity - the mean squared displacement of the end-to-end distance (MSD of ETE)
further referred as MSD.

\subsubsection{Comparison to free chain}

In this section, the investigation focuses on understanding 
the influence of anchoring on chain dynamics. 
This is achieved by comparing the MSD of a fully-flexible anchored 
chain with the predictions provided by the Rouse model. 
By contrasting simulation observations with theoretical
expectations, insights are gained into how anchoring 
affects chain dynamics.

\begin{figure}[h]
    \begin{center}
      \includegraphics[width=\columnwidth,trim={0cm 0cm 0cm 0.9cm},clip]{3-exp-fixed-param-log.png}
      \caption{\label{fig:anchored_flex_chain_vs_rouse}
      MSD of ETE of anchored full-flexible chain vs predictions of rouse model (free chain, Eq. \ref{eq:rouse_msd_ete}).
      Filled area corresponds MSD curve $\pm$ 3 standard deviations of the mean. The
      line connecting data points and the filled area between data points doesn't make
      any statements about probability of measuring values in this interval and is
      added for readability.
      }
    \end{center}
\end{figure}

Figure \ref{fig:anchored_flex_chain_vs_rouse} shows, that by anchoring the chain
the the transition into plateau is shifted to the right, increasing the 
relaxation time (Eq. \ref{eq:rouse_relaxation_time}), 
however the scaling behavior is visually the same. Rouse relaxation time
of the free chain predicted using rouse model is: $\tau_R=130.16$, $\tau_0=0.941$. 
The rouse relaxation time of the anchored chain estimated 
using the fit of the Eq. \ref{eq:rouse_msd_ete} with $\tau_R$ as free parameter
to the MSD curve of acnhored chain is: 
$\tau_R=582.3 \pm 28.6$, $\tau_0=4.21 \pm 0.07$, which is approximately $4.47$ times
larger. The fitted curve is displayed on Fig.
\ref{fig:anchored_flex_chain_vs_rouse_fitted}.
Intuitively such difference is clear, the beginning of the anchored chain
can't move and therefore the chain needs more time to achieve the MSD limit, which
in case of fully flexible free chain is $2Nl_b^2$ as the one can see from Eq. \ref{eq:rouse_msd_ete}. 


\begin{figure}[h]
    \begin{center}
      \includegraphics[width=\columnwidth,trim={0cm 0cm 0cm 0.8cm},clip]{3-exp-free-param-log.png}
      \caption{\label{fig:anchored_flex_chain_vs_rouse_fitted}
      MSD of ETE of anchored full-flexible chain vs fit of rouse model prediction 
      (Eq. \ref{eq:rouse_msd_ete}) with $\tau_R$ as free parameter.
      Filled area corresponds MSD curve $\pm$ 3 standard deviations of the mean. The
      blue line connecting data points and the filled area between data points doesn't make
      any statements about probability of measuring values in this interval and is
      added for readability.
      }
    \end{center}
\end{figure}

It is possible to introduce correction factor based on the estimated $\tau_R$
to account for boundary conditions:
\begin{equation}
    \alpha := \frac{\tau_{0, \textrm{empirical}}}{\tau_{0, \textrm{analytical}}} \approx 4.47
\end{equation}



\subsubsection{Impact of chain stiffness}
Within this subsection, the focus shifts 
toward exploring the influence of chain stiffness on
the dynamics of anchored polymer chains. 
By manipulating the stiffness of the chain, the investigation aims to 
unveil the intricate interplay between molecular rigidity 
and the effects of anchoring. The analysis provides insights into 
how varying the chain stiffness exert an influence on the
conformational transitions and dynamics exhibited by anchored chains.
Table \ref{table:kappa_values}. shows the range of stiffness values examined
in this experiment.
\\
\\
The MSD curves are plotted in Figure \ref{fig:msd_anchored_l_K}.
The following observations are made:
\begin{itemize}
    \item The relaxation time grows non-linearly with rising $l_K$
    \item Long time MSD limit grows non-linearly with rising $l_K$,
    however for $l_K/L >= 0.65$ it is not possible to distinguish 
    the curves any more because of the uncertainty.
\end{itemize}

\begin{figure}[h]
    \begin{center}
      \includegraphics[width=\columnwidth,trim={0cm 0cm 0cm 0.0cm},clip]{4-exp-delta_R-bare.png}
      \caption{\label{fig:msd_anchored_l_K}
      MSD of ETE of anchored full-flexible chain vs fit of rouse model prediction 
      (Eq. \ref{eq:rouse_msd_ete}) with $\tau_R$ as free parameter.
      Filled area corresponds MSD curve $\pm$ 3 standard deviations of the mean. The
      blue line connecting data points and the filled area between data points doesn't make
      any statements about probability of measuring values in this interval and is
      added for readability.
      }
    \end{center}
\end{figure}

\begin{figure}[h]
    \begin{center}
      \includegraphics[width=\columnwidth,trim={0cm 0cm 0cm 0.0cm},clip]{4-exp-delta_R-bare-log.png}
      \caption{\label{fig:msd_anchored_l_K}
      MSD of ETE of anchored full-flexible chain vs fit of rouse model prediction 
      (Eq. \ref{eq:rouse_msd_ete}) with $\tau_R$ as free parameter.
      Filled area corresponds MSD curve $\pm$ 3 standard deviations of the mean. The
      blue line connecting data points and the filled area between data points doesn't make
      any statements about probability of measuring values in this interval and is
      added for readability.
      }
    \end{center}
\end{figure}

Moreover, the attempt is made to describe this results analytically. At first
the results are compared to the Rouse model predictions for the chain
with $N_K$ segments of length $l_K$. The results of this comparison are shown
in Figure \ref{fig:msd_anchored_l_K_rouse_fit_anal}. It is clear, that
Rouse model is not able to explain the dynamics of the semi-flexible chains.
The reason for that is violation of contunious chain assumption, because
the small number of chain segments.

\begin{figure}[h]
    \begin{center}
      \includegraphics[width=\columnwidth,trim={0cm 0cm 0cm 0.0cm},clip]{4-exp-delta_R-rouse_anal.png}
      \caption{\label{fig:msd_anchored_l_K_rouse_fit_anal}
      MSD of ETE of anchored full-flexible chain vs fit of rouse model prediction 
      (Eq. \ref{eq:rouse_msd_ete}) with $\tau_R$ as free parameter.
      Filled area corresponds MSD curve $\pm$ 3 standard deviations of the mean. The
      blue line connecting data points and the filled area between data points doesn't make
      any statements about probability of measuring values in this interval and is
      added for readability.
      }
    \end{center}
\end{figure}

Further, the MSD in main-axis coordinate system is analyzed.
Figure \ref{fig:msd_anchored_l_K_by_dim} shows the MSD curves in main axis system
for each dimension. The analysis delivers following insights:
\begin{itemize}
    \item The difference of MSD in $z$ dimension relative to the $x$ and $y$
    dimensions rises with growing stiffness. The plateau value of MSD falls.
    \item The MSD in $z$ dimension has smaller relaxation time in case $l_K/L \ge 0.95$ 
\end{itemize}
The chain becomes more straight with rising stiffness and therefore has less
movement freedom along $z$ direction, which results in smaller MSD and quicker
relaxation.

\begin{figure}[h]
    \begin{center}
      \includegraphics[width=\columnwidth,trim={0cm 0cm 0cm 0.0cm},clip]{4-exp-msd_by_dim.png}
      \caption{\label{fig:msd_anchored_l_K_by_dim}
      MSD of ETE of anchored full-flexible chain vs fit of rouse model prediction 
      (Eq. \ref{eq:rouse_msd_ete}) with $\tau_R$ as free parameter.
      Filled area corresponds MSD curve $\pm$ 3 standard deviations of the mean. The
      blue line connecting data points and the filled area between data points doesn't make
      any statements about probability of measuring values in this interval and is
      added for readability.
      }
    \end{center}
\end{figure}

\begin{table}[h]
    
    \centering
\begin{tabular}{rr}
    \toprule
    $\kappa$ & $l_K / L$ \\
    \midrule
    1.00 & 0.03 \\
    11.00 & 0.33 \\
    21.00 & 0.65 \\
    31.00 & 0.97 \\
    41.00 & 1.29 \\
    51.00 & 1.60 \\
    61.00 & 1.92 \\
    71.00 & 2.24 \\
    \bottomrule
    \end{tabular}
    \caption{
        Values of $\kappa$ and corresponding $l_K$ tried in the study
        of anchored chain dynamics.
        }
    \label{table:kappa_values}
\end{table}

\subsubsection{Impact of friction coefficient of chain end}

\subsection{Free chain dynamics}




% Erklärung
\clearpage
\thispagestyle{empty}
\minisec{Erklärung}\vspace*{1.5em}

Hiermit erkläre ich, dass ich diese Arbeit im Rahmen der Betreuung am Institut
für ??? Physik ohne unzulässige Hilfe Dritter verfasst und alle Quellen als solche gekennzeichnet habe.

\vspace*{45em}

Vorname Nachname \par
Dresden, Monat 2019

\bibliographystyle{plain}
\bibliography{sources}

\end{document}
