\documentclass[handout]{beamer}
\usepackage{graphicx}
\usepackage{caption}
\usepackage{subcaption}
\graphicspath{ {./img/} }
\usetheme{default}

\title{Molecular dynamics study of ideal polymer chains with variable persistence length}
\subtitle{Short progress report}
\author{Yahor Paromau, Holger Merlitz}
\institute{ITP@IPF}
\date{23.05.2023}

\newcommand{\mean}[1]{\langle #1 \rangle}

\begin{document}

\begin{frame}
    \titlepage
\end{frame}

\begin{frame}
    \frametitle{Outline}
    \tableofcontents
\end{frame}

\begin{frame}
    \frametitle{Definitions, notations, units}
    Mostly: following Svaneborg article \cite{svaneborg_2020}

    Units: LJ

    Notations:

    \begin{itemize}
        \item Contour length: $L$
        \item End to End distance (ETE): $\vec{R}$
        \item Change of ETE: $(\Delta R(t))^2 := [\vec{R}(t)-\vec{R}(0)]^2$
        \item Friction coefficient, viscosity: $\zeta$ $[\frac{\textrm{mass}}{\textrm{time}}]$, $\eta$ $[\frac{\textrm{mass}}{\textrm{time} * \textrm{distance}}]$
        \item subscript "$b$" to denote bead specific properties to distinguish these from Kuhn units:
            \begin{itemize}
                \item Kuhn lenght, bond length: $l_K$, $l_b$
                \item Number of Kuhn segments, number of beads: $N_K$, $N_b$
            \end{itemize}
        \item Rouse relaxation time \cite{svaneborg_2020}: $\tau_R = \frac{1}{3 \pi^2} \frac{\zeta_{CM} \mean{R^2}}{k_B T} = \frac{1}{3 \pi^2} \frac{\zeta N_b^2 l_b^2}{k_B T}$
        \item Relaxation time of single bead: $\tau_0 = \frac{\tau_R}{N^2}$ 
    \end{itemize}

\end{frame}
    
\begin{frame}
    \frametitle{Assumptions}
    \begin{itemize}
        \item Variation of $0.2\%$ of $l_b$ is neglectible \cite{svaneborg_2020} 
        
        $\Rightarrow$ $l_b=const$, $L=(N_b-1) l_b$=const

        \item ...
    \end{itemize} 
\end{frame}

\begin{frame}
    \frametitle{Rouse model}
    \begin{equation}
        \mean{R^2}=N_b l_b^2
    \end{equation}
    \begin{equation}
        g_4(t) := \mean{(\Delta R(t))^2} = 2 N_b l_b^2 (1-\frac{8}{\pi^2}\sum_{p=1,3,...}e^{\frac{-t p^2}{\tau_R}})
    \end{equation}

\end{frame}

\begin{frame}
    \frametitle{Experiment 1: Fully flexible chain $l_K=l_b$}

\end{frame}

\section{References}

\setbeamertemplate{bibliography item}{\insertbiblabel}
\begin{frame}
    \frametitle{References}
    \bibliographystyle{IEEEtran}
    \bibliography{refs}
\end{frame}

\end{document}